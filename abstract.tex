% Advances in modern data collection technologies have led to a proliferation of complex, structured and high-dimensional objects, such as images and curves, for data analysis.
% A common first step in statistical modelling is to learn a latent feature representation of these objects, i.e., a transformation to a lower-dimensional Euclidean space of latent features.

% {\color{red} add downsides of current work?}

We present GLaRe, a software tool to assess the performance of latent feature representation methods.
GLaRe computes and summarizes the full cross-validated distribution of information losses for a latent feature representation method and can be used to select among different methods to represent a dataset.
We apply GLaRe to three motivating datasets to choose between principal components analysis, autoencoder and wavelet representations.
The three case studies demonstrate that different representations are suitable for different datasets and hence the utility of GLaRe in finding an optimal representation of a given dataset.
GLaRe is available as a \proglang{R} package and can be downloaded from \proglang{GitHub}\footnote{\url{https://github.com/edwardgunning/GLaRe}}.
