Advances in modern data collection technologies have led to a proliferation of complex, structured and high-dimensional objects, such as images and curves, for data analysis.
A common first step in statistical modelling is to learn a latent feature representation of these objects, i.e., a transformation to a lower-dimensional Euclidean space of latent features.

{\color{red} add downsides of current work?}

In this work, we present \pkg{GLaRe}, a software tool that can be used to assess the performance of different latent feature representation methods.
\pkg{GLaRe} computes cross-validated measure of information loss for a latent feature representation, produces graphics to summarise the full distribution of information losses, and can be used to choose between different methods on a given dataset.
We demonstrate the use of \pkg{GLaRe} on three motivating datasets, where we use it to choose between an principal components analysis, an autoencoder and the discrete wavelet transform.
The three case studies demonstrate that different representations are most suitable for different datasets and hence the utility of \pkg{GLaRe} for finding an optimal representation for a given dataset.
\pkg{GLaRe} is available as a \proglang{R} package and can be downloaded at \url{https://github.com/edwardgunning/GLaRe}.
