\section{Introduction}

Advancements in computer storage capabilities and computational speed have made high-dimensional complex data ubiquitous in all areas of science.
Efficient analysis of such data often necessitates a lower-dimensional representation in a ``latent" space that retains the salient structure of the original data and is more amenable to statistical modeling.
We use the term \emph{latent feature representation method} to refer to statistical and machine learning approaches that achieve this dimension reduction through a (linear or non-linear) transformation of the data to a lower-dimensional space of features.
Examples of latent feature representation methods include Principal Component Analysis (PCA) \parencite{hotelling_analysis_1933}, wavelet representations \parencite{daubechies_wavelet_1990}, t-distributed stochastic neighbor embedding (t-SNE) \parencite{maaten_visualizing_2008}, uniform manifold approximation and projection (UMAP) \parencite{mcinnes_umap_2020} and autoencoders \parencite{rumelhart_learning_1986}.

These latent representations are routinely used in downstream analysis, e.g., latent features can be employed as predictors in multivariable regression or classification, in clustering, or as the response vector in multivariate regression models \parencite{niu_dimensionality_2011,wang_role_2014, cook_fisher_2007}. 
Therefore, assessing how well these methods preserve information is a pertinent challenge.
Training error (i.e., how well the method reconstructs the training data) provides a naive estimate of information loss that tends to be overly-optimistic.
\emph{Generalization error}, which can be defined as a latent feature representation method's error in reconstructing unseen data (i.e., data not used to learn the representation), is used to more accurately quantify information loss and is typically computed using cross-validation approaches \parencite[see, e.g.,][]{becht_dimensionality_2019, bro_cross-validation_2008, wold_cross-validatory_1978, eastment_cross-validatory_1982,krzanowski_cross-validation_1987, minka_automatic_2000, rajan_bayesian_1994, camacho_cross-validation_2014, diana_cross-validation_2002, hubert_fast_2007, josse_selecting_2012, saccenti_use_2015}.

However, most existing approaches summarize generalization error using a single statistic that is aggregated across all observations (e.g., average or total information loss) and reflects the central tendency of the full distribution of individual losses, which can mask variation across observations.
For example, a satisfactory average performance might hide cases where individual observations are very poorly represented and if a generative statistical model is formulated for the latent features, it might disproportionately favor observations that are reconstructed well.
To ensure that models formulated in the latent space can reflect the true data-generating process, it is important to evaluate the entire distribution of generalization errors and use control metrics such as the worst-case performance or quantiles of the generalization error distribution.
In addition, evaluation of a method's information preservation must be balanced with the compactness of the representation, as prioritizing losslessness alone can lead to unnecessarily complex representations, whereas overly compact (``lossy") representations may fail to preserve important dataset characteristics.

In this work, we present \textbf{CLaRe} (compact near-lossless latent feature representations), a framework designed to assess and select among latent feature representations using the full distribution of generalization errors.
Our approach uses cross-validation to compute this distribution and uses it to evaluate a latent feature representation using a coherent set of user-specified criteria.
The key contributions of our framework are as follows:
\begin{enumerate}
    \item In contrast to conventional approaches focusing on aggregated measures that reflect the central tendency of the error distribution (training or generalization error), the CLaRe framework ensures that a level of error tolerance is met for quantiles of the distribution of generalization error (e.g., worst case or $95$th percentile). Latent feature representation methods are evaluated on their ability to preserve all (or most) of the salient information in a dataset.
    \item The suitability of a latent feature representation depends heavily on the characteristics of the dataset at hand \parencite[Section 3, pp. 325--328]{morris_functional_2015}.
    By defining a consistent set of criteria to evaluate different latent feature representation methods, CLaRe enables objective comparisons among different methods to identify the most suitable latent representation method for a specific dataset and application.
    As we demonstrate in our case studies, this facilitates comparisons between traditional statistical tools like PCA and modern machine learning approaches such as autoencoders.
    \item The framework is accompanied by a user-friendly software implementation in our \proglang{R} package called \pkg{GLaRe} (Graphical Analysis of Latent Representations).
    The package provides graphical summaries to aid the selection of an optimal latent feature representation.
    It provides built-in latent feature representation methods as well as the option for the user to provide their own bespoke method.
\end{enumerate}
We demonstrate the practical utility of CLaRe through case studies on three high-dimensional datasets from diverse fields of application: 1) measurements of the mechanical strain at different locations in the eye, with application to the study of Glaucoma \parencite{lee_bayesian_2019}, 2) a neurobiological dataset of gel images of proteins from the brains of rats, with application to drug-use addiction \parencite{morris_pinnacle_2008}, and 3) the well-known MNIST dataset of handwritten digits that is widely used image recognition \parencite{lecun_mnist_1998}.
These case studies highlight that different datasets often favor different latent feature representation methods, emphasizing the importance of a consistent, objective and flexible evaluation framework.


The remainder of this article is structured as follows.
In Section \ref{sec:materials-and-methods} we present the methodological foundations of latent feature representations and generalization error that underpin CLaRe. 
In Section \ref{sec:case-studies}, we introduce three motivating datasets and employ the CLaRe framework to assess the performance of PCA, Discrete Wavelet Transform (DWT) and autoencoder representations of these datasets and select an optimal representation of each.
In Section \ref{sec:software}, we document the software implementation of our framework, GLaRe (Graphical Analysis of Latent Representations).
We close with a discussion in Section \ref{sec:discussion}.