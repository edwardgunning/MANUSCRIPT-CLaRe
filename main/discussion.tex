\section{Discussion}\label{sec:discussion}

We have presented \pkg{GLaRe}, a new software tool for assessing different latent feature representation methods.
Our tool advances the existing software and literature in three main ways.
First, \pkg{GLaRe} places a unique focus on estimating the full distribution of information loss, which is more sensitive and informative than summary or total measures.
We have coined terminology, e.g., tolerance level, cut-off criterion and qualifying criterion, top characterise this distribution and use it to assess a latent feature representation method.
Second, \pkg{GLaRe} estimates information loss through cross-validation, which ensures generlaisation error is accurately capured.
Finally, \pkg{GLaRe} is not tied to any latent feature representation method and can be used to compare among several methods, as demonstrated in our case studies.
We have presented a detailed overview of the terminology and methodology that underpins GLaRe, a description of the software functionality and output, and the results of three case studies where we use GLaRe to select among different latent feature representation methods for our three motivating datasets. 
The results of our case studies emphasise the utility of \pkg{GLaRe}, as each of the motivating datasets favoured a different latent feature representation method.
We briefly discuss some limitations and future directions below.


Limitations:
\begin{itemize}
    \item Didn't optimise AE and DWT efficiency/ architecture.
\end{itemize}

Extensions:
\begin{itemize}
    \item More built in methods (CNN, VAE...), functional PCA.
    \item Other loss functions.
    \item Dependencies in data. Time series cross validation. Structured/ multilevel models. Leave-one-subject-out cross-validation. \url{https://nsojournals.onlinelibrary.wiley.com/doi/full/10.1111/ecog.02881}.
    \item GUI and RShiny.
    \item Other wrapper functions and summaries.
    \item Parrlaell.
\end{itemize}