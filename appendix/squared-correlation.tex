\section{Squared Correlation as a Loss Measure}

All things done for centered data. Emma does some proofs for non-centered but not sure the mean used for centering is consistent between paper and software.

\subsection{Predicted Residual Sum of Squares (PRESS)}

The predicted residual sum of squares (PRESS) statistics measures the discrepancy, in terms of total squared error, between the observed data $\mathbf{X}$ and its predicted value $\widehat{\mathbf{X}}^{(K)} = f^{(-1)}_K(\mathbf{X}^*)$. For a given row of $\mathbf{X}$, which we denote by $X_{i\cdot}$, the PRESS statistic is defined as
$$
\text{PRESS}\left(X_{i\cdot}, \widehat{X}^{(K)}_{i\cdot}\right) = \sum_{t = 1}^T \left(X_{it} - \widehat{X}^{(K)}_{it}\right)^2 = \left(X_{i\cdot} - \widehat{X}^{(K)}_{i\cdot}\right)^\top \left(X_{i\cdot} - \widehat{X}^{(K)}_{i\cdot}\right).
$$

\subsection{Predicted Correlation Squared ($R^2$)}

The predicted correlation denoted $R^2$ measures the correlation among the observed data $\mathbf{X}$ and its predicted value $\widehat{\mathbf{X}}^{(K)} = f^{(-1)}_K(\mathbf{X}^*)$. It is defined as
$$
R^2 \left(X_{i\cdot}, \widehat{X}^{(K)}_{i\cdot}\right) =
\frac{\left(\sum_{t = 1}^T X_{it}\widehat{X}^{(K)}_{it}\right)^2}{\sum_{t = 1}^T X_{it}^2\sum_{t = 1}^T\left(\widehat{X}^{(K)}_{it}\right)^2}.
$$

\subsection{Relationship Between PRESS and $R^2$ for PCA}

We expand PRESS and $R^2$ to better express the relationship between them, when PCA is employed as the latent feature representation method. 
We start by recalling that, when PCA is applied to the mean-centered matrix $\mathbf{X}$, the $K$-dimensional prediction is given by 
$$
\widehat{X}^{(K)}_{i\cdot} = \underbrace{\boldsymbol{\Phi}_K}_{\text{PCs}} \underbrace{\boldsymbol{\Phi}_K^\top X_{i\cdot}}_{\text{PC Scores}}
$$
\begin{align*}
    \text{PRESS}\left(X_{i\cdot}, \widehat{X}^{(K)}_{i\cdot}\right) &= \left(X_{i\cdot} - \widehat{X}^{(K)}_{i\cdot}\right)^\top \left(X_{i\cdot} - \widehat{X}^{(K)}_{i\cdot}\right) \\
    &= 
    X_{i\cdot}^\top X_{i\cdot} - 2 X_{i\cdot}^\top \widehat{X}^{(K)}_{i\cdot} + \widehat{X}^{(K)\top}_{i\cdot} \widehat{X}^{(K)}_{i\cdot} \\
    &= X_{i\cdot}^\top X_{i\cdot} - 2 X_{i\cdot}^\top \widehat{X}^{(K)}_{i\cdot} + X_{i\cdot}^\top \boldsymbol{\Phi}_K \underbrace{\boldsymbol{\Phi}_K^\top \boldsymbol{\Phi}_K}_{= \mathbf{I}_K} \boldsymbol{\Phi}_K^\top X_{i\cdot} \\
    &=  X_{i\cdot}^\top X_{i\cdot} - 2 X_{i\cdot}^\top \widehat{X}^{(K)}_{i\cdot} + X_{i\cdot}^\top \underbrace{\boldsymbol{\Phi}_K \boldsymbol{\Phi}_K^\top X_{i\cdot}}_{=\widehat{X}^{(K)}_{i\cdot}} \\
    &= X_{i\cdot}^\top X_{i\cdot} - 2 X_{i\cdot}^\top \widehat{X}^{(K)}_{i\cdot} + X_{i\cdot}^\top \widehat{X}^{(K)}_{i\cdot} \\
    &= X_{i\cdot}^\top X_{i\cdot} - X_{i\cdot}^\top \widehat{X}^{(K)}_{i\cdot}.
\end{align*}
Likewise, we have 
\begin{align*}
    R^2 \left(X_{i\cdot}, \widehat{X}^{(K)}_{i\cdot}\right) 
    &=
    \left(\widehat{X}^{(K)^\top}_{i\cdot}X_{i\cdot} X_{i\cdot}^\top \widehat{X}^{(K)^\top}_{i\cdot} \right)
     \left(X_{i\cdot}^\top X_{i\cdot} \underbrace{\widehat{X}^{(K)\top}_{i\cdot}\widehat{X}^{(K)}_{i\cdot}}_{=X_{i\cdot}^\top \widehat{X}^{(K)}_{i\cdot}} \right)^{-1} \\
     &= 
     \left(\widehat{X}^{(K)^\top}_{i\cdot}X_{i\cdot} X_{i\cdot}^\top \widehat{X}^{(K)^\top}_{i\cdot} \right)
     \left(X_{i\cdot}^\top X_{i\cdot} X_{i\cdot}^\top \widehat{X}^{(K)}_{i\cdot} \right)^{-1} \\
     &=
     \widehat{X}^{(K)^\top}_{i\cdot}X_{i\cdot} \underbrace{X_{i\cdot}^\top \widehat{X}^{(K)^\top}_{i\cdot}
     \left(X_{i\cdot}^\top \widehat{X}^{(K)^\top}_{i\cdot} \right)^{-1}}_{=1}
     \left(X_{i\cdot}^\top X_{i\cdot}\right)^{-1}  \\
     &= \widehat{X}^{(K)^\top}_{i\cdot}X_{i\cdot}  \left(X_{i\cdot}^\top X_{i\cdot}\right)^{-1} \\
     &=
     \widehat{X}^{(K)^\top}_{i\cdot}X_{i\cdot}  \left(X_{i\cdot}^\top X_{i\cdot}\right)^{-1} + 
     \left(X_{i\cdot}^\top X_{i\cdot}\right)  \left(X_{i\cdot}^\top X_{i\cdot}\right)^{-1} -
     \left(X_{i\cdot}^\top X_{i\cdot}\right)  \left(X_{i\cdot}^\top X_{i\cdot}\right)^{-1} \\
     &= 
     1 - \underbrace{\left( X_{i\cdot}^\top X_{i\cdot} - \widehat{X}^{(K)^\top}_{i\cdot}X_{i\cdot} \right)}_{= \text{PRESS}\left(X_{i\cdot}, \widehat{X}^{(K)}_{i\cdot}\right)} \left(X_{i\cdot}^\top X_{i\cdot}\right)^{-1} \\
     &=
     1 - \text{PRESS}\left(X_{i\cdot}, \widehat{X}^{(K)}_{i\cdot}\right) \left(X_{i\cdot}^\top X_{i\cdot}\right)^{-1}.
\end{align*}
When calculate the \emph{total} statistics over all observations, we have an analogous relations:
\begin{align*}
     \text{PRESS}\left(\mathbf{X}, \widehat{\mathbf{X}}^{(K)}\right) &= \sum_{i=1}^N \left(X_{i\cdot} - \widehat{X}^{(K)}_{i\cdot}\right)^\top \left(X_{i\cdot} - \widehat{X}^{(K)}_{i\cdot}\right) \\
     &= \sum_{i=1}^N X_{i\cdot}^\top X_{i\cdot} - X_{i\cdot}^\top \widehat{X}^{(K)}_{i\cdot}.
\end{align*}
\begin{align*}
    R^2\left(\mathbf{X}, \widehat{\mathbf{X}}^{(K)}\right)
    &= \frac{\left(\sum_{i=1}^N \sum_{t=1}^T X_{it} \widehat{X}_{it}^{(K)}\right)^2}{
    \left(\sum_{i=1}^N \sum_{t=1}^T X_{it} X_{it}\right)
    \left(\sum_{i=1}^N \sum_{t=1}^T \widehat{X}_{it}^{(K)} \widehat{X}_{it}^{(K)}\right)
    } \\
    &= 
    \frac{\left(\sum_{i=1}^N \widehat{X}^{(K)^\top}_{i\cdot}X_{i\cdot} \right)^2}{
    \left(\sum_{i=1}^N \sum_{t=1}^T X_{it} X_{it}\right)
    \left(\sum_{i=1}^N \sum_{t=1}^T \widehat{X}_{it}^{(K)} \widehat{X}_{it}^{(K)}\right)
    } \\
\end{align*}

When PRESS is calculated \emph{in sample}, we have
When calculate the \emph{total} statistics over all observations, we have an analogous relations:
\begin{align*}
     \text{PRESS}\left(\mathbf{X}, \widehat{\mathbf{X}}^{(K)}\right) &= \sum_{i=1}^N \left(X_{i\cdot} - \widehat{X}^{(K)}_{i\cdot}\right)^\top \left(X_{i\cdot} - \widehat{X}^{(K)}_{i\cdot}\right) \\
     &= \sum_{i=1}^N X_{i\cdot}^\top X_{i\cdot} - X_{i\cdot}^\top \widehat{X}^{(K)}_{i\cdot} \\
     &= \text{need to show } \sum_{r=1}^R \widehat{\lambda}_r  - \sum_{r=K+1}^R \widehat{\lambda}_r  \\
     &= \sum_{r=1}^K \lambda_r
\end{align*}