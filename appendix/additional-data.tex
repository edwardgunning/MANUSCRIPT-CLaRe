\section{Additional Data Example: The \texttt{phoneme} Dataset}\label{sec:additional-data}

The \texttt{phoneme} dataset, from the book \emph{Nonparametric Functional Data Analysis: Theory and Practice} by \textcite{ferraty_nonparametric_2006}\footnote{Data available at \url{https://www.math.univ-toulouse.fr/~ferraty/SOFTWARES/NPFDA/}}, is a dataset from the field of speech recognition analysis.
It emanates from an original dataset used in the book \emph{The Elements of Statistical Learning} by \textcite{hastie_elements_2009}\footnote{Data available at \url{https://hastie.su.domains/ElemStatLearn/}}.
The data contains observations of an audio signal that is transformed to the log-periodogram scale at a range of frequencies.
\textcite{ferraty_nonparametric_2006} provide $N=2000$ observations of the signal discretized onto a grid of $T=150$ equidistant frequencies, so that $\mathbf{X}$ is a $2000\times 150$ matrix containing the signals in its rows.
Figure \ref{fig:phoneme} displays a random sample of $8$ observations from the dataset.

\begin{figure}[h]
    \centering
    \includegraphics[width=0.75\linewidth]{figures/phoneme.pdf}
    \caption{A random sample of $8$ observations from the \texttt{phoneme} dataset \parencite{hastie_elements_2009, ferraty_nonparametric_2006}.}
    \label{fig:phoneme}
\end{figure}

Figure \ref{fig:phoneme-results} displays the results of applying our CLaRe framework to select among PCA, DWT and AE representations for the phenome dataset. A grid of equally-spaced values from $1$ to $150$ in increments of $5$ was used for the latent feature dimensions.
The qualifying criterion was achieved for PCA and DWT but not for the AE. The qualifying dimensions for PCA and DWT were $qd=126$ and $qd = 146$, respectively. Hence, PCA was the favored latent feature representation method for this dataset.

\begin{figure}
    \centering
    \includegraphics[width=1\linewidth]{figures/phoneme-results.pdf}
    \caption{Summary \texttt{GLaRe()} plot for the \texttt{phoneme} data. A grid of equally-spaced values from $1$ to $150$ in increments of $5$ was used for the latent feature dimensions.}
    \label{fig:phoneme-results}
\end{figure}