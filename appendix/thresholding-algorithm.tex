\section{Wavelet Thresholding Algorithm}

We use the Discrete Wavelet Transform (DWT) algorithm implementation in the \texttt{dwt()} function from the \pkg{wavselim} \proglang{R} package.
Our thresholding approach is described below, and demonstrated on the \texttt{DTI} dataset from the \pkg{refund} \proglang{R} package \parencite{goldsmith_refund_2020}.

\begin{steps}
  \item \underline{\textbf{Pad the Data to Dyadic Length}}: The DWT can only be applied to vectors of dyadic length, i.e., a power of $2$. In most cases, the we work with the $N \times T$ data matrix $\mathbf{X}$ where $T$ is not a power of $2$ (i.e., $\log_2(T)$ is not an integer). If this is the case, we define $\log_2(T_{pad})$ as the smallest integer greater than $\log_2(T)$. We then add $\lfloor (T_{pad} - T)/2 \rfloor$ columns of $0$'s to the left and $\lceil (T_{pad} - T)/2 \rceil$ columns of $0$'s to the right side of $\mathbf{X}$, so that the resulting matrix $\mathbf{X}_{pad}$ has dimensions $N \times T_{pad}$ (Figure \ref{fig:DTI-padded}).
  \begin{figure}[H]
      \centering
      \includegraphics[width=0.75\linewidth]{figures/DTI-padded.pdf}
      \caption{Padding the \texttt{DTI} data to transform it from length $T = 93$ to $T_{pad} = 128 = 2^7$.}
      \label{fig:DTI-padded}
  \end{figure}
  \item \underline{\textbf{Apply the DWT to Each Row}}: We then apply the DWT to each row of $\mathbf{X}_{pad}$, which transforms each vector from $T_{pad}$ measurements of a time series (or signal) to $T_{pad}$ wavelet coefficients. 
  We add store the wavelet coefficients for each row in the rows of the $N \times T_{pad}$ matrix $\mathbf{X}^*$.
  When we have expanded the original signal by padding in Step 1, we can expect a number of the $T_{pad}$ wavelet coefficients to be $0$, however this number is likely to be less than $T_{pad} - T$.
  \item \underline{\textbf{Compute the Relative Energy Matrix}}: For each row of $\mathbf{X}^*$, we have the vector of wavelet coefficients $\mathbf{X}^*_{i\cdot} = (X^*_{i1}, \dots,X^*_{iT_{pad}})^\top$. We denote the \emph{Total Energy} for the $i$th observation as the sum of its squared wavelet coefficients
  $$\text{Total Energy}_i = \sum_{k=1}^{T_{pad}}X^{*2}_{ik}.$$ Next, we define the \emph{Cumulative Relative Energy} for the $i$th observation and wavelet coefficient $k$ as 
  $$
  \text{Relative Energy}_{ik} = \frac{\sum_{\{k: \lvert X^*_{ik'}\rvert  \geq \lvert X^*_{ik}\rvert \}}X^{*2}_{ik}}{\text{Total Energy}_i}.
  $$
  This quantity represents the proportion of the total energy that is explained by the $k$th wavelet coefficient and all coefficients greater in absolute value than it. Hence, smaller values indicate this coefficient is important and values closer to $1$ indicate less importance (i.e., a value of $1$ indicates that all of the energy has been explained before this coefficient).
  Normalising by the total energy is important because we summarise this quantity across all $i$ as a measure of importance in the next step, and the normalisation ensures that it the importance is not obscured by the total energy of an individual signal. We let $\textbf{En}^*$ represent the total energy matrix which contains $\text{Relative Energy}_{ik}$ in its $i$th row and $k$th column.
  \item \underline{\textbf{Compute the Relative Energy Scree}}: To summarise the overall importance of each of the wavelet coefficients we average each column of the matrix $\textbf{En}^*$. We obtain the $T_{pad}$-dimensional \emph{Scree} vector, that has the $k$th entry
  $$
  \text{Scree}_k = \frac{1}{N} \sum_{i=1}^N \text{Relative Energy}_{ik}.
  $$
  As with the individual relative energy matrix, coefficients with a lower average value are of greater importance while larger average values (closer to $1$) indicate less importance.
  \item \underline{\textbf{Hard Thresholding Based on the Relative Energy Scree}}: For a given $K < T_{pad}$, we threshold the wavelet coefficient matrix $\mathbf{X}^*$ based on the relative energy scree. That is, we retain the $K$ columns of $\mathbf{X}^*$ that have the smallest values of $\text{Scree}_k$ and set the remaining columns to $0$. We denote the thresholded version of $\mathbf{X}^*$ by $\widehat{\mathbf{X}}^{*(K)}$.
  \item \underline{\textbf{Apply IDWT to Each Row of the Thresholded Coefficient Matrix}}: To transform back to the data space, we apply the inverse DWT (IDWT) to each row of $\widehat{\mathbf{X}}^{*(K)}$. This will give use the reconstructed matrix $N\timesT_{\pad}$
  $$
  \widehat{\mathbf{X}}^{(K)}_{pad} = \text{IDWT}(\widehat{\mathbf{X}}^{*(K)}).
  $$
  To obtain a representation of the original signal length we simply discard the first $\lfloor (T_{pad} - T)/2 \rfloor$ columns and the last $\lceil (T_{pad} - T)/2 \rceil$ columns to give the matrix $\widehat{\mathbf{X}}^{(K)}$.
\end{steps}
