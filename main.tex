\documentclass[12pt]{article}
\usepackage{xcolor} % for different colour text
\usepackage{tikz} % for figures
\usepackage[utf8]{inputenc}
\usepackage{mathpazo}
\usepackage[section]{placeins}
\usepackage{multirow}
\usepackage[normalem]{ulem}
\usepackage{relsize}
\usepackage[ruled]{algorithm2e}
\usepackage{graphicx}
\usepackage{booktabs}
\usepackage{framed}
\usepackage{listings}
\usepackage{float}
\usepackage{lipsum}
\usepackage{mathtools, amsfonts}
\usepackage{siunitx}
\usepackage{amsmath, amssymb} % for bold sybmols
\usepackage{bm} % for bold symbols
\usepackage[acronym]{glossaries} % for abbreviations
\usepackage{enumitem}
\newlist{steps}{enumerate}{1}
\setlist[steps, 1]{label = Step \arabic*:}
%\usepackage{bbold} % for identity matrix sign
\usepackage[colorlinks,allcolors=black,citecolor=blue,urlcolor=blue]{hyperref}
\usepackage[citestyle=apa,style=apa,backend=biber,url=false,uniquename=false, useprefix=true, doi=false]{biblatex}
\AtEveryBibitem{%
  \clearfield{note}%
  \clearfield{urlyear}
  \clearfield{urlmonth}
}
\usepackage{dsfont} % for fancy R
\addbibresource{references.bib} % Imports bibliography file
\DefineBibliographyStrings{english}{%
  circa = {{}ca\adddot},
}

% setting (should go in Style sheet:)
\newcommand{\identity}{\mathbf{I}}
\newcommand{\fundmat}{\vec{\Phi}(t, 0)}
\newcommand{\fundmats}{\vec{\Phi}(t, s)}
\newcommand{\fundmatso}{\vec{\Phi}(s, 0)}
\newcommand{\fundmatsu}{\vec{\Phi}(s, u)}
\newcommand{\fundmattv}{\vec{\Phi}(t, v)}
\newcommand{\xtilde}{\widetilde{\vec{x}}}
\lstdefinestyle{mystyle}{  
    commentstyle=\color{codegreen},
    keywordstyle=\color{magenta},
    numberstyle=\tiny\color{gray},
    stringstyle=\color{codepurple},
    basicstyle=\ttfamily\footnotesize,
    breakatwhitespace=false,         
    breaklines=true,                 
    captionpos=b,                    
    keepspaces=true,                 
    numbers=left,                    
    numbersep=5pt,                  
    showspaces=false,                
    showstringspaces=false,
    showtabs=false,                  
    tabsize=2
}

\lstset{style=mystyle}

% for algorithm 2e
\SetKwComment{Comment}{/* }{ */}
% ---------------------------------------------------------------------------
% New commands
\newcommand{\1}{\mathbf{1}}
\newcommand{\boldbeta}{\boldsymbol{\beta}}
\newcommand{\boldut}{\mathbf{u}_i(t)}
\newcommand{\boldepst}{\boldsymbol{\varepsilon}_{ij}(t)}
\newcommand{\Expec}{\mathbb{E}}

\newcommand{\Var}{\operatorname{Var}}
\newcommand{\SE}{\operatorname{SE}}
\newcommand{\Cov}{\operatorname{Cov}}
\newcommand{\diag}{\operatorname{diag}}
\newcommand{\ICC}{\operatorname{ICC}}
\DeclareMathOperator{\vect}{vec}

\newcommand{\add}[1]{\textcolor{blue}{#1}}
\newcommand{\delete}[1]{\textcolor{red}{\sout{#1}}}
\newcommand{\edit}[2]{\textcolor{red}{\sout{#1}} \textcolor{blue}{#2}}

% 
\newcommand{\pkg}[1]{{\normalfont\fontseries{b}\selectfont #1}} \let\proglang=\textsf \let\code=\texttt

%To make placeins use subsection instead of section%
\makeatletter
\AtBeginDocument{%
  \expandafter\renewcommand\expandafter\subsection\expandafter
    {\expandafter\@fb@secFB\subsection}%
  \newcommand\@fb@secFB{\FloatBarrier
    \gdef\@fb@afterHHook{\@fb@topbarrier \gdef\@fb@afterHHook{}}}%
  \g@addto@macro\@afterheading{\@fb@afterHHook}%
  \gdef\@fb@afterHHook{}%
}
\makeatother

% ------- %
\setlength{\textwidth}{16cm}
\setlength{\textheight}{22cm}
\setlength{\hoffset}{-1.4cm}
\topmargin -1cm
% ------- %




\title{GLaRe: A Graphical Tool for Assessing
Losslessness of Latent Feature Representations}
\author{
Emma Zohner\thanks{Department of Statistics, Rice University.}
\and
Edward Gunning\thanks{Department of Biostatistics, Epidemiology and Informatics, University of Pennsylvania.}
\and
Giles Hooker\thanks{Department of Statistics and Data Science, University of Pennsylvania.} 
\and
Jeffrey Morris\thanks{Department of Biostatistics, Epidemiology and Informatics, University of Pennsylvania.}
}
\date{}


\begin{document}

\maketitle



\begin{abstract}
% Advances in modern data collection technologies have led to a proliferation of complex, structured and high-dimensional objects, such as images and curves, for data analysis.
% A common first step in statistical modelling is to learn a latent feature representation of these objects, i.e., a transformation to a lower-dimensional Euclidean space of latent features.

% {\color{red} add downsides of current work?}

Latent feature representation methods play an important role in the dimension reduction and statistical modeling of high-dimensional complex object data.
Existing approaches to assess how well these methods preserve information have typically been limited to a single statistic, which is aggregated over all observations (e.g., total or average loss) and can mask individual observations being represented poorly.
We propose GLaRe, a framework and software tool for characterizing the full distribution of generalization error for a latent feature representation method.
We use cross-validation to compute the full distribution of generalization error and we present a framework for using this distribution to select among different latent feature representations for a given dataset.
Our associated software tool implements the framework and provides graphical summaries that can be used to aid the analysis.
We apply GLaRe to three motivating datasets to select among principal components, wavelet representations, or autoencoders.
The three case studies demonstrate that different representations are suitable for different datasets and hence the utility of GLaRe in finding an optimal representation of a given dataset.
Our software is available as an \proglang{R} package and can be downloaded from \proglang{GitHub}\footnote{\url{https://github.com/edwardgunning/GLaRe}}.




% We present GLaRe, a software tool to assess the performance of latent feature representation methods \jsm{for high-dimensional complex object data}.
% GLaRe computes and summarizes the full cross-validated distribution of information losses for a {\jsm linear or nonlinear} latent feature representation method\jsm{,} can be used to select among different methods to represent a dataset \jsm{and choose a latent feature dimension to accomplish a desired level of near-losslessness}

\end{abstract}

\clearpage

\tableofcontents

\clearpage
\section{Introduction}

Advancements in computer storage capabilities and computational speed have made high-dimensional complex data ubiquitous in all areas of science. A common approach for effciently analyzing and modeling these complex objects is to find a lower-dimensional representation that retains the salient characteristics of the data without a significant information loss.
We use the term \emph{latent feature representation method} to refer to statistical and machine learning approaches that achieve this dimension reduction through a (linear or non-linear) transformation of the data to a lower-dimensional space of features.
Examples of feature representation methods include principal components analysis (PCA), independent component analysis (ICA), canonical correlation analysis (CCA), isomaps, non-negative matrix factorization (NMF), linear discriminant analysis (LDA), sparse coding, wavelet transform, t-distributed stochastic neighbor embedding (t-SNE), uniform manifold approximation and projection (UMAP) and auto-encoders. 
These methods aim to retain the dominant characteristics of data in a lower-dimensional structure, thereby reducing its dimension. 
In practice, one computes the lower-dimensional representation, then applies standard statistical and machine learning tools to the new features. 
For example, these new features can be used as predictors in multivariable regression or classification, in clustering, or as the response vector in multivariate regression models.

The latent representation space often possesses properties for modelling that are more amenable to statistical modelling than the data space, such as
\begin{itemize}
    \item \emph{\underline{Sparsity/ Compression}}: The latent representation space can be of dimension much smaller than the observed data. A compact representation can reduce storage requirements and computational effort in downstream analysis. For example, dimension reduction is used extensively in image compression to minimize the size of graphics files while keeping the quality of the image to an acceptable level \parencite{marcellin_overview_2000}.
    \item \emph{\underline{Regularisation}}: Models and algorithms applied to the latent features often benefit in performance from a reduction of noise and collinearity in high-dimensional data. For example, dimension reduction has been shown to improve results in clustering \parencite{niu_dimensionality_2011}, classifcation \parencite{wang_role_2014} and regression \parencite{cook_fisher_2007}.
    \item \emph{\underline{Visualisation and Interpretation}}: Lower-dimensional latent representations often facilitate intuitive and interpretable visualisations of high dimensional data structures, e.g., for curve or single-cell data objects \parencite{jones_displaying_1992, maaten_visualizing_2008, hyndman_rainbow_2010, becht_dimensionality_2019}.
\end{itemize}

Since these lower-rank representations of the data are used in downstream analyses, any information that is lost in the latent feature representation step is lost in all subsequent analyses. 
Therefore, it is important to quantify how much information is retained in the representation. 
% In fact, the key question in latent feature representation techniques is how well new features capture the information in the data.
Assessing information loss on the same data used to learn the representation will typically result in estimates that overly optimistic.
\emph{Generalization error}, which can be defined as a latent feature representation method's error in reconstructing unseen data, can be used to answer this question and accurately quantify information loss in latent feature representations. 
There is a rich literature on analytical and empirical generalization error for dimension reduction, with a particular focus on cross-validation approaches for PCA \parencite[see, e.g.,][]{becht_dimensionality_2019, wold_cross-validatory_1978, eastment_cross-validatory_1982,krzanowski_cross-validation_1987, minka_automatic_2000, rajan_bayesian_1994, camacho_cross-validation_2014, diana_cross-validation_2002, hubert_fast_2007, josse_selecting_2012, saccenti_use_2015}.

However, in these papers and more generally in practice, generalization error for latent feature representations is expressed as a single statistic that summarises the distribution of generalization errors of individual observations e.g., the average or total error.
Often a single number is not sufficient to characterise the distribution of individual generalization errors, e.g., a respectable average generalization error might mask the fact that a number of observations are being represented poorly. 
To be confident that a given latent feature representation captures the information and structure in an entire dataset, its performance over the full collection of observations needs to be assessed so that features of the generalisation error distribution can be evaluated, e.g., the median, the best case, the worst case and other quantiles.

It is also well established that the suitability of a latent feature representation depends heavily on the characteristics of the dataset at hand, i.e., there is no preferred ``one size fits all" method.
For example, this has been emphasised in the context of basis function representations for functional data -- curves with different smoothness, regularity and sampling characteristics are suited to different types of basis functions, e.g., wavelets, splines and functional principal components \textcite[Section 3, pp. 325--328]{morris_functional_2015}.
However, most available methods for computing generalization error focus on a single latent feature representation method (e.g., PCA) and do not allow comparison between methods by using similar statistics or summaries.

In this article, we introduce a Graphical Latent Feature Representation tool (GLaRe) for assessing how well a learned latent feature representation captures the salient information in a given dataset, allowing the user to select an optimal representation for their data.
The tool is developed as an \proglang{R} \parencite{r_core_team_r_2022} package and is accessible at \proglang{GitHub}\footnote{\url{https://github.com/edwardgunning/GLaRe}}. 
Although there is existing software for dimension reduction, which we briefly survey below, the tool that we present is unique in its simultaneous focus on assessing the generalization error for individual observations and facilitating comparisons between different methods.
\textcite{samudrala_software_2014} developed software for dimension reduction that provides a graphical user interface that outputs an optimal low-dimensional representation, but it does not address generalization of the dimension reduction methods, and it is specifically tailored to chemical crystallography data.
\textcite{zubova_dimensionality_2018} compared the speed and accuracy of dimension reduction and provided visualisations, but generalization error and individual-level performance measures were not provided. 
An \proglang{R} package for a single method, Multifactor Dimensionality Reduction, was developed by \textcite{winham_r_2011}. Visualization of the quality of dimension reduction, as well as point-wise quality measures are discussed by \textcite{mokbel_visualizing_2013} but no software is provided. 
In \textcite{cavallo_visual_2018} a visual interaction framework is presented for dimension reduction where users can directly manipulate and modify data through dimension reduction visualizations. This framework's focus is on interaction via forward and backward projection to improve the use of dimensionality
reduction.
To summarise, there is a lack of user-friendly software to assess the performance of different latent feature representations at the individual observation level.

The remainder of this article is structured as follows.
In Section \ref{sec:materials-and-methods}, we present the methodological foundations of latent feature representations and generalisation error, and we introduce three motivating datasets that are used as a case study in this article. 
In Section \ref{sec:software}, we document our novel Graphical Latent Feature Representation (GLaRe) tool.
Section \ref{sec:results} presents the results of our case study, where we use our proposed Graphical Latent Feature Representation (GLaRe) tool to assess the performance of PCA, the Discrete Wavelet Transform (DWT) and and autoencoder representations of our three motivating datasets.
We close with a discussion in Section \ref{sec:discussion}.


\section{Materials and Methods}\label{sec:materials-and-methods}

\subsection{Motivating Datasets}\label{sec:motivating-datasets}
A data object is a collection of information. Types of data objects include collections of images, documents, spatiotemporal recordings, or genomics information. 
Data objects are often sampled and structured into a table or matrix for analysis, modeling and inference. 
We present three data objects that motivate our method of computing a proper latent feature representation:

{\color{purple}Need to be more clear that we are working with functonal data.}

\subsubsection{Glaucoma Data}




\subsubsection{Proteomic Gels Data}


\subsubsection{MNIST Digits Data}

\begin{figure}
    \centering
    \includegraphics[width=1\textwidth]{example-image-c}
    \caption{
    A sample observation from each of our three motivating datasets.
    \textbf{(a)}: A sample glaucoma image, representing a polar azimuthal projection of MPS functionsfor a single eye at one IOP level.
    \textbf{(b)}: A sample 2D gel electrophoresis image, showing proteomic content in the brain tissue of a rat.
    \textbf{(c)}: A sample MNIST digit image, which is a $28 \times 28$ pixel greyscale image of single handwritten digit.}
    \label{fig:enter-label}
\end{figure}

\subsection{Latent Feature Representations}

\add{Suppose that we have $N$ observations of object data, denoted by $X_1 (t), \dots, X_N(t)$, where $t$ indexes a location on a continuous domain $\mathcal{T}$ over which the objects are defined.
For time-varying curves, $\mathcal{T}$ is generally a closed subset of the real line that represents a (normalized) time interval.
However, as exemplified in our three motivating datasets, the domain $\mathcal{T}$ can be multi-dimensional to represent locations in an image or surface, and it can also be non-Euclidean ({see Glaucoma data/ \textcite{lee_bayesian_2019}}).
We assume that each observation is measured on a common\footnote{In practice, the measurement grids of individual observations need not be identical if they are all sufficiently fine such that interpolation onto a common, fine grid is feasible.}, ordered grid of $T$ points in $\mathcal{T}$, denoted by $\mathbf{t} = \left(t_1, \dots, t_T\right)^\top$, and we let $X_i(\mathbf{t}) = \left(X_i(t_1), \dots, X_i(t_T)\right)^\top$.
Then, we can represent the observed data in the $N \times T$ data matrix $\mathbf{X}$, which contains the vectors $X_1(\mathbf{t}), \dots, X_N(\mathbf{t})$ in its rows.
We refer to the $T$-dimensional space of features in which the observed data are represented as the \emph{data space}.}

\add{We define a \emph{latent feature representation} as a method that transforms each observation from the data space\footnote{If we have have representation of the underlying object $X_i(t)$ that does not involve measurement on a grid, the transformation can be defined on the object itself as $f_{K} \left(X_i(t)\right) = \left(X_{i1}^*, \dots,  X_{iK}^* \right)^\top$.} to a new space of latent features, called the \emph{representation space}.
Mathematically, we define the latent feature representation of the $i$th observation as
$$
f_{K} \left(X_i(\mathbf{t})\right) = \left(X_{i1}^*, \dots,  X_{iK}^* \right)^\top,
$$
where the number of features $K$ defines the dimensionality of the representation space and can range between $1$ and some possible maximum $K_{max}$.
When $K \ll T$, we say that the latent feature representation is \emph{sparse}.
As we expand on in Section {\color{purple}X}, the dimension of the representation space is generally selected according to some criteria of information loss.
We also require that there exists an inverse transformation $f^{-1}_K$ that maps each observation from the representation space back to the data space as
$$
\widehat{X}_i(\mathbf{t}) = f_{K}^{-1} \left( \left(X_{i1}^*, \dots,  X_{iK}^* \right)^\top \right).
$$}

\add{Linear transformations of the form $f_{K} \left(X_i(\mathbf{t})\right) = \mathbf{A} X_i(\mathbf{t})$, for some $K \times T$ transformation matrix $\mathbf{A}$, are often used in practice.
For example, it is common to represent a functional observation $X_i(t)$ as a linear combination of a set of basis functions $\{\phi_k(t)\}_{k=1}^K$, which defines the inverse transformation
$$
\widehat{X}_i(\mathbf{t}) = \sum_{k=1}^K X_{ik}^* \phi_k(\mathbf{t}) = \boldsymbol{\Phi} \left(X_{i1}^*, \dots,  X_{iK}^* \right)^\top,
$$
where $\boldsymbol{\Phi} = \left[\phi_1(\mathbf{t}) | \dots | \phi_K(\mathbf{t}) \right]$ and the latent features $X_{ik}^*$ are basis coefficients. 
When these basis coefficients are computed by ordinary least squares, the linear transformation matrix defining the latent feature representation $f_K$ is of the form $\mathbf{A} = \left( \boldsymbol{\Phi}^\top \boldsymbol{\Phi} \right)^{-1} \boldsymbol{\Phi}^\top$.
When the matrix of basis function evaluations $\boldsymbol{\Phi}$ is orthogonal, $\mathbf{A} = \boldsymbol{\Phi}^\top$, i.e., the transformation $f_K$ is simply right multiplication by this matrix.
However, in general, there is no need for the transformation $f_K$ to be orthogonal or even linear, and non-linear transformations may be preferred for certain types of data.}


\subsection{Assessing Losslessness via Generalization Error}

\add{Although modelling is performed in the representation space due to its attractive properties, we often want to transform modelling results back to the data space for inference, interpretation and visualisation. 
As such, the accuracy and interpretation of an analysis depends on the degree of information that is preserved when moving between the data and representation spaces for a given latent feature representation.
In what follows, we characterise the degree of information loss of a latent feature representation on a full dataset.
}



\subsection{Characterising Information Loss}
We characterise the degree of information loss of a latent feature representation for each individual observation as
$$
\text{Loss} \left( f_K(X_i(t)) \right) = \lVert  X_i(\mathbf{t}) - f^{-1}_{K}(X_{ik}^*) \rVert,
$$
where $\lVert \boldsymbol{\cdot} \rVert$ denotes a measure (e.g., $\mathcal{L}_2$ norm) such that $\lVert  X_i(\mathbf{t}) - f^{-1}_{K}(X_{ik}^*) \rVert$ defines a distance between $X_i(\mathbf{t})$ and $f^{-1}_{K}(X_{ik}^*)$ \parencite{morris_comparison_2017}.
We say that the transformation $f_K$ is \emph{lossless} for the $i$th observation $X_i(t)$ if
$$
\text{Loss} \left( f_K(X_i(\mathbf{t})) \right) = 0,
$$
and lossless for the full dataset $X_1(t), \dots, X_N (t)$ if
$$
\text{Loss} \left( f_K(X_i(\mathbf{t})) \right) = 0 \quad \forall \quad  i = 1, \dots, N.
$$
That is, we only refer to a latent feature representation as lossless for a given dataset if the representation is near lossless for every individual observation in that dataset.
More generally, we can allow some tolerance of information loss and say that
the transformation $f_K$ is \emph{near-lossless} for the $i$th observation $X_i(t)$ if
$$
\text{Loss} \left( f_K(X_i(\mathbf{t})) \right) < \epsilon,
$$
for a chosen tolerance level $\epsilon$. 
Similarly, we say that the transformation $f_K$ is near-lossless for the full dataset only if each individual observation achieves this tolerance level, that is
$$
\text{Loss} \left( f_K(X_i(\mathbf{t})) \right) < \epsilon \quad \forall \quad  i = 1, \dots, N.
$$
In this case, it is important to note that the distinction between this definition and a measure such as the average of individual losses $\frac{1}{N}\sum_{i=1}^N \text{Loss} \left( f_K(X_i(\mathbf{t})) \right)$.
For example, Figure \ref{fig:ind-losses} displays the distribution of individual information losses on a dataset sample dataset of a PCA latent of varying dimensions.
In this case, we are using the squared correlation measure as our loss, so a value $0$ means the representation captures no information and a value of $1$ means that the representation is lossless.
The grey points represent the individual observations' losses, whereas the red squares indicate the average loss.
The figure highlights the information that can be hidden when on a single measure is used to describe the full distribution of losses. For example, at $k = 1$ the average loss is at $0.4$ but there are observations with individual losses at almost $0$.


\begin{figure}
    \centering
    \includegraphics[width=0.75\textwidth]{figures/info-loss.pdf}
    \caption{Phoneme data \textcite{hastie_elements_2009}. \add{\textbf{To Do: Trace minimum, maximum and quantiles.}}}
    \label{fig:ind-losses}
\end{figure}
\newline

It often suffices to achieve this tolerance level for the majority of observations, i.e., there may be a small number of observations for which achieving near-losslessness is not possible. We can extend our definition to accommodate this notion by introducing a \emph{qualifying criterion (qc)}, which is the proportion of observations in the dataset that achieve near-losslessness at a given tolerance level.
Mathematically, we can define the transformation $f_K$ as near-lossless at a tolerance level $\epsilon$ and a qualifying criterion $qc$\footnote{Could try to write as $\frac{\sum_{i \text{ s.t. } \text{Loss} \left( f_K(X_i(\mathbf{t})) \right) < \epsilon }1}{N}$?}
$$
\frac{| \{x \in \{X_i(\mathbf{t})\}_{i}^N : \text{Loss} \left( f_K(x) \right) < \epsilon \} |} 
{N}
\geq qc,
$$
where the notation $|S|$ represents the cardinality (i.e., number of elements) in a set $S$.}
\subsubsection{Generalization Error of Individual Data Objects}
\subsubsection{Performance Across Quantiles of Individual Data Objects}
\section{Software}\label{sec:software}

The main function in the \pkg{GLaRe} software package is \texttt{GLaRe()}, which performs and summarises the cross-validated information of a given latent feature method on a given dataset.
A sample call to the \texttt{GLaRe()} function is as follows:
\begin{lstlisting}[language=R]
mnist_pca <- GLaRe(mat = x_train,
                   latent_dim_from = 1,
                   latent_dim_to = 400,
                   latent_dim_by = 20,
                   learn = "pca",
                   verbose = TRUE)
\end{lstlisting}
The \texttt{GLaRe()} function computes cross-validated measures of information loss for a latent feature representation method defined by \texttt{learn} on dataset stored in the matrix \texttt{mat} across a range of latent feature dimensions defined by a grid of equally-spaced valiues from \texttt{latent\_dim\_from} to \texttt{latent\_dim\_by} in increments of \texttt{latent\_dim\_by}.

\subsection{Learning Functions}

The learning function is the main engine of \texttt{GlaRe()}.
The learning function function takes as arguments a data matrix $\mathbf{X}$ and latent feature dimension $K$, and learns the encoding and decoding transformation functions $f_K$ and $f_K^{-1}$.
For example, the learning function in PCA simply computes the first $K$ eigenvectors of the empirical covariance matrix of $\mathbf{X}$, and $f_K$ and $f_K^{-1}$ comprise the matrix multiplication by the first $K$ eigenvectors.
In contrast, for an autoencoder, $f_K$ and $f_K^{-1}$ are general functions that map to and from a $K$-dimensional space and are parametrised by flexible neural networks.
In \texttt{GLaRe()}, the learning function defined by the \texttt{learn} argument. We provide three built-in learning functions that can be used in the software, as well as the option to specify a user-defined bespoke function:
\begin{enumerate}
    \item Setting \texttt{learn = "pca"} specifies a PCA representation. The eigenvectors are computed by the Singular Value Decomposition (SVD) algorithm.
    \item Setting \texttt{learn = "dwt"} (or \texttt{learn = "dwt.2d"} for data on a 2-dimensional domain) specifies a thresholded wavelet representation. For encoding, the DWT is applied to $\mathbf{X}$ and the most important $K$ latent features (i.e., wavelet coefficients) are learned from the data and are retained. The decoding function then applies the inverse DWT to the retained features. Our implementation uses the \pkg{wavselim} \proglang{R} package \parencite{whitcher_waveslim_2024} which uses the Daubechies orthonormal compactly supported wavelet of length $8$ \parencite{daubechies_ten_1992}, least asymmetric family and uses periodic boundary conditions. Additional details are described in Appendix \ref{sec:wavelet-thresholding-algorithm}.
    \item Setting \texttt{learn = "ae"} specifies an autoencoder representation. We implement the autoencoder using the \pkg{keras} \proglang{R} package \parencite{kalinowski_keras_2024}. The encoder and decoder functions are parametrised by neural networks with two hidden layers (defaulting to sizes of $600$ and $200$, respectively) and rectified linear unit (ReLu) activation functions.
    A linear activation is used to map from the second hidden layer of the encoder to the latent space, and a choice between linear and sigmoid (default) activation functions can be used to map from the second hidden layer of the decoder back to the latent space.
    By default, the autoencoder is trained for $100$ epochs using the ADAM stochastic gradient descent algorithm \parencite{kingma_adam_2017} to minimise either the mean squared error (default) or binary cross-entropy loss functions using a mini-batch size of $16$.
    \item Setting \texttt{learn = "user"} allows the user to specify their own latent feature representation method. With this setting, the user must supply the learning function for their method, that takes the data matrix $\mathbf{X}$ and the latent feature dimension $\mathbf{K}$ as inputs and returns a list with two elements: functions named \texttt{Encode} and \texttt{Decode} implementing the the learned encoding and decoding transformation functions $f_K$ and $f_K^{-1}$.
\end{enumerate}

\subsection{Squared Correlation Loss}

In principle, any loss function can be used with \texttt{GLaRe()} and the package has been structured such that different loss functions can be used in future iterations.
The current implementation uses the compliment of the squared correlation
$$
1- \rho^2 \left(X_i (\mathbf{t}), \widehat{X}^{(K)}_{i} (\mathbf{t}) \right) =
1 - \frac{\left( \mathlarger{\sum}_{t = 1}^T{\bigg(X_i (t) - \overline{X}_i \bigg) \bigg( \widehat{X}_i^{(K)} (t) - \overline{\widehat{X}}_i^{(K)} \bigg)} \right)^2}{\mathlarger{\sum}_{t = 1}^T \bigg(X_i (t) - \overline{X}_i \bigg)^2 \mathlarger{\sum}_{t = 1}^T \bigg( \widehat{X}_i^{(K)} (t) - \overline{\widehat{X}}_i^{(K)} \bigg) ^2},
$$
where
$$
\overline{X}_i = \frac{1}{N} \sum_{t=1}^T X_i (t) \quad \text{and} \quad \overline{\widehat{X}}_i^{(K)} = \frac{1}{N} \sum_{t=1}^T \widehat{X}_i^{(K)} (t).
$$
As a measure of information loss, $1- \rho^2$ is quick and easy to compute.
It is a relative measure, bounded between $0$ and $1$: $1- \rho^2 = 0$ indicates losslessness and $1- \rho^2 = 1$ indicates that no information was preserved by the latent feature representation method.
In Appendix \ref{sec:squared-correlation}, we highlight connections between squared correlation, the Predicted Residual Sum of Squares (PRESS) loss and the scree plot for PCA.

\subsection{Cross-Validation}

\subsection{Software Outputs}

The \texttt{GLaRe()} function computes and returns the cross-validated information loss for each individual observation in the dataset.
The distribution of these independent values is summarized by the main plot that is returned by default by the \texttt{GLaRe()} function (Figure {\color{red}X}).
The overall, or average, cross-validated loss is displayed in yellow, with the analogous loss computed on the training data shown in green for comparison.
Then, different quantiles of the distribution of individual cross-validated are displayed to summarise the full distribution: the minimum and maximum are shown in blue and red, respectively, a user-specified quantile of the distribution (set by the \texttt{cvqlines} argument) is displayed in purple and the quantile of the distribution being used as the cut-off criterion $\alpha$ (defaulting to 0.95, i.e., the $95$th percentile) is displayed in light grey.
The corresponding value of the tolerance level $\epsilon$ is overlaid as grey dashed horizontal line, and hence the latent feature dimension (i.e., location on the $x$-axis) at which the two grey lines meet corresponds to the qualifying criterion.
The \texttt{GLaRe()} function also returns a heatmap as an alternative summary of the individual cross-validated loss distribution, and additional wrapper functions in the package produce a dotplot of the full distribution, a function that plots the ratio of training to cross-validated losses and functions to display the reconstructions of individual observations. Further details of additional functionalities are provided in Appendix \ref{sec:additional-outputs}.

\begin{figure}
    \centering
    \includegraphics[width=0.5\linewidth]{}
    \caption{Anatomy of the \texttt{GlaRe()} summary plot applied to the {\color{red}X} data.}
    \label{fig:enter-label}
\end{figure}
\section{Results}\label{sec:results}

\subsection{Glaucoma Data}
\subsection{Proteomic Gels Data}

\begin{figure}
    \centering
    \includegraphics[width=1\linewidth]{figures/initial-gels.pdf}
    \caption{Preliminary results for the gels data.}
    \label{fig:enter-label}
\end{figure}
\subsection{MNIST Digits Data}
\section{Discussion}\label{sec:discussion}

We have presented GLaRe, a new software tool for assessing different latent feature representation methods.
We have presented a detailed overview of the terminology and methodology that underpins GLaRe, a description of the software functionality and output, and the results of three case studies where we use GLaRe to select among different latent feature representation methods for our three motivating datasets.

GLaRe places a unique focus on estimating the full distribution of generalisation error for a given dataset, which is more sensitive and informative than summary or total measures.
We have coined new terminology, e.g., ``tolerance level", ``cut-off criterion" and ``qualifying criterion", to characterise this distribution and use it to assess latent feature representation methods.
Our software produces a summary plot as well as several other visualisations to summarise and characterise this distribution.

A key feature of GLaRe is that it is not tied to any latent feature representation method and can be used to compare among several methods, as demonstrated through our case studies.
The results of these case studies emphasise the utility of GLaRe, as each of the motivating datasets favoured a different latent feature representation method.
They also re-enforce that sample size plays (alongside data structure) an aimportant when choosing a latent feature representation method.
For the MNIST dataset, the sample size ($N=60000$) is large relative to the feature dimension ($T=784$) so it was possible to estimate a flexible non-linear transformation of the data using the AE that provided an optimal representation of the data.
On the other hand, the Proteomic Gels data has a small sample size ($N=53$) relative to the feature dimension ($T=556206$) so the fixed DWT representation was preferred to the more flexible PCA and AE representations.
We performed an experiment by manually decreasing the sample size of the Glaucoma dataset and comparing the PCA and DWT representations to further re-enforce this point.

Some limitations and future directions of this work are as follows.
Firstly, we focused on providing standard, rather than specialised, implementations of PCA and the AE.
While it is possible to use the \texttt{learn = "user"} setting to specify user-defined transformations, e.g., smoothed functional PCA for curve data or convolutional AEs for image data, a future direction is to implement them along the current PCA, DWT and AE options.
Likewise, we favoured the squared correlation loss due to its inherent links to conventional loss quantities in PCA (Appendix \ref{sec:squared-correlation}), but our software is structured such that future iterations can use alternative other loss functions (e.g., concordance index {\color{red}ref. Yang}) that are specified or provided by the user.







Extensions:
\begin{itemize}
    \item Dependencies in data. Time series cross validation. Structured/ multilevel models. Leave-one-subject-out cross-validation. \url{https://nsojournals.onlinelibrary.wiley.com/doi/full/10.1111/ecog.02881}.
    \item GUI and RShiny.
    \item Other wrapper functions and summaries.
    \item Parrlaell.
\end{itemize}

%----------------------------%
\section*{Acknowledgements}
An early draft of this work first appeared as part of Emma Zohner's Ph.D. dissertation (\citeyear{zohner_feature_2021}, Chapter 2).

% \section*{Supplementary Implementation}

% The \pkg{GLaRe} \proglang{R} package and a repository containg the code and data to reproduce all examples in this paper are available on \proglang{GitHub}\footnote{\url{https://github.com/edwardgunning/GLaRe}}\footnote{\url{https://github.com/edwardgunning/MANUSCRIPT-GLaRe-A-Graphical-Tool-for-Assessing-Losslessness-of-Latent-Feature-Representations}}.
\clearpage

\appendix
\section{Additional Software Outputs and Functionality}\label{sec:additional-outputs}


The \texttt{GLaRe()} function returns a heatmap to display the full distribution of generalization errors (Figure \ref{fig:eye-heatmap}).
It is obtained by re-ordering the $N$ values within each column of the matrix of cross-validated information losses.
The latent feature dimension is represented on the $x$-axis, the corresponding quantile of the generalization error distribution at that feature dimension (i.e., column) is shown on the $y$-axis and the color represents the value of the generalization error at that feature dimension and quantile.


\begin{figure}
    \centering
    \includegraphics[width=0.75\linewidth]{figures/eye-heatmap.png}
    \caption{The heatmap returned by \texttt{GLaRe()} used to summarize the full distribution of generalization errors (i.e., cross-validated estimates of information loss). The latent feature dimension is represented on the $x$-axis, the corresponding quantile of the generalization error distribution at that feature dimension is shown on the $y$-axis and the color represents the value of the generalization error at that feature dimension and quantile.}
    \label{fig:eye-heatmap}
\end{figure}

The \pkg{GLaRe} package also contains wrapper functions that plot alternative summaries of the cross-validated distribution of information losses; their outputs are displayed in Figure \ref{fig:additional-plots-01}.
The function \texttt{distribution\_plot()} produces a dot-plot of the individual cross-validated information loss distribution, where each point represents an individual value and the points are colored according to the latent feature dimension $K$ (Figure \ref{fig:additional-plots-01} \textbf{(a)}).

\begin{figure}
    \centering
    \includegraphics[width=1\linewidth]{figures/additional-plots-01.pdf}
    \caption{Additional wrapper functions that produce summary plots of \texttt{GLaRe()} outputs. \textbf{(a)} \texttt{distribution\_plot()} produces a dot-plot of the individual cross-validated information loss distribution. \textbf{(b)} \texttt{plot\_train\_validation\_ratio()} produces a point and line plot of the ratio of the total training and validation losses.
    Both plots are demonstrated on the Glaucoma dataset with a PCA representation from Section \ref{sec:software}.}
    \label{fig:additional-plots-01}
\end{figure}

When the qualifying criterion is met, the CoLLaRe framework re-trains the final model on the full dataset at the qualifying dimension.
The \pkg{GLaRe} package contains functions to visually inspect the reconstruction of individual observations from the final model.
Due to the non-standard structure of the Glaucoma, Proteomic Gels and MNIST data, we have written custom functions to display side-by-side plots of the data observation and its reconstruction (Figures \ref{fig:eye-reconstruction} -- \ref{fig:mnist-reconstruction}).
For data objects which are $1$-dimensional signals, we provide a general function called \texttt{plot\_1D\_reconstruction()} that displays the original signal as a solid line and overlays its reconstruction as a dotted line (Figure \ref{fig:phoneme-reconstruction}).
Unlike the specialized functions that plot side-by-side plots, this function can display the reconstruction of more than one observation simultaneously.

\begin{figure}
    \centering
    \includegraphics[width=0.75\linewidth]{figures/eye-reconstruction.pdf}
    \caption{A single observation from the Glaucoma data (left) and its reconstruction (right) using the final model fit from PCA at the qualifying dimension $qd = 51$. The figure was generated using the \texttt{plot\_eye\_reconstruction()} function from the \pkg{GLaRe} package.}
    \label{fig:eye-reconstruction}
\end{figure}

% \begin{figure}
%     \centering
%     \includegraphics[width=0.75\linewidth]{}
%     \caption{A single observation from the Proteomic Gels data (left) and its reconstruction (right) using the final model fit from DWT at the qualifying dimension $qd = 7801$. The figure was generated using the \texttt{plot\_gels\_reconstruction()} function from the \pkg{GLaRe} package.}
%     \label{fig:gels-reconstruction}
% \end{figure}

\begin{figure}
    \centering
    \includegraphics[width=0.75\linewidth]{figures/mnist-reconstruction.pdf}
    \caption{A single observation from the MNIST Digits data (left) and its reconstruction (right) using the final model fit from AE at the qualifying dimension $qd = 101$. The figure was generated using the \texttt{plot\_mnist\_reconstruction()} function from the \pkg{GLaRe} package.}
    \label{fig:mnist-reconstruction}
\end{figure}

\begin{figure}
    \centering
    \includegraphics[width=0.75\linewidth]{figures/phoneme-reconstruction.pdf}
    \caption{Reconstructions of $8$ observations from the \texttt{phenome} dataset (dotted lines) overlaid on the original observations (solid lines). The reconstructions were computed from the final model fit of PCA at the qualifying dimension $qd=126$. The figure was generated using the \texttt{plot\_1D\_reconstruction()} function from the \pkg{GLaRe} package.}
    \label{fig:phoneme-reconstruction}
\end{figure}
\section{Wavelet Thresholding Algorithm}

We use the Discrete Wavelet Transform (DWT) algorithm implementation in the \texttt{dwt()} function from the \pkg{wavselim} \proglang{R} package.
Our thresholding approach is described below, and demonstrated on the \texttt{DTI} dataset from the \pkg{refund} \proglang{R} package \parencite{goldsmith_refund_2020}.

\begin{steps}
  \item \underline{\textbf{Pad the Data to Dyadic Length}}: The DWT can only be applied to vectors of dyadic length, i.e., a power of $2$. In most cases, the we work with the $N \times T$ data matrix $\mathbf{X}$ where $T$ is not a power of $2$ (i.e., $\log_2(T)$ is not an integer). If this is the case, we define $\log_2(T_{pad})$ as the smallest integer greater than $\log_2(T)$. We then add $\lfloor (T_{pad} - T)/2 \rfloor$ columns of $0$'s to the left and $\lceil (T_{pad} - T)/2 \rceil$ columns of $0$'s to the right side of $\mathbf{X}$, so that the resulting matrix $\mathbf{X}_{pad}$ has dimensions $N \times T_{pad}$ (Figure \ref{fig:DTI-padded}).
  \begin{figure}[H]
      \centering
      \includegraphics[width=0.75\linewidth]{figures/DTI-padded.pdf}
      \caption{Padding the \texttt{DTI} data to transform it from length $T = 93$ to $T_{pad} = 128 = 2^7$.}
      \label{fig:DTI-padded}
  \end{figure}
  \item \underline{\textbf{Apply the DWT to Each Row}}: We then apply the DWT to each row of $\mathbf{X}_{pad}$, which transforms each vector from $T_{pad}$ measurements of a time series (or signal) to $T_{pad}$ wavelet coefficients. 
  We add store the wavelet coefficients for each row in the rows of the $N \times T_{pad}$ matrix $\mathbf{X}^*$.
  When we have expanded the original signal by padding in Step 1, we can expect a number of the $T_{pad}$ wavelet coefficients to be $0$, however this number is likely to be less than $T_{pad} - T$.
  \item \underline{\textbf{Compute the Relative Energy Matrix}}: For each row of $\mathbf{X}^*$, we have the vector of wavelet coefficients $\mathbf{X}^*_{i\cdot} = (X^*_{i1}, \dots,X^*_{iT_{pad}})^\top$. We denote the \emph{Total Energy} for the $i$th observation as the sum of its squared wavelet coefficients
  $$\text{Total Energy}_i = \sum_{k=1}^{T_{pad}}X^{*2}_{ik}.$$ Next, we define the \emph{Cumulative Relative Energy} for the $i$th observation and wavelet coefficient $k$ as 
  $$
  \text{Relative Energy}_{ik} = \frac{\sum_{\{k: \lvert X^*_{ik'}\rvert  \geq \lvert X^*_{ik}\rvert \}}X^{*2}_{ik}}{\text{Total Energy}_i}.
  $$
  This quantity represents the proportion of the total energy that is explained by the $k$th wavelet coefficient and all coefficients greater in absolute value than it. Hence, smaller values indicate this coefficient is important and values closer to $1$ indicate less importance (i.e., a value of $1$ indicates that all of the energy has been explained before this coefficient).
  Normalising by the total energy is important because we summarise this quantity across all $i$ as a measure of importance in the next step, and the normalisation ensures that it the importance is not obscured by the total energy of an individual signal. We let $\textbf{En}^*$ represent the total energy matrix which contains $\text{Relative Energy}_{ik}$ in its $i$th row and $k$th column.
  \item \underline{\textbf{Compute the Relative Energy Scree}}: To summarise the overall importance of each of the wavelet coefficients we average each column of the matrix $\textbf{En}^*$. We obtain the $T_{pad}$-dimensional \emph{Scree} vector, that has the $k$th entry
  $$
  \text{Scree}_k = \frac{1}{N} \sum_{i=1}^N \text{Relative Energy}_{ik}.
  $$
  As with the individual relative energy matrix, coefficients with a lower average value are of greater importance while larger average values (closer to $1$) indicate less importance.
  \item \underline{\textbf{Hard Thresholding Based on the Relative Energy Scree}}: For a given $K < T_{pad}$, we threshold the wavelet coefficient matrix $\mathbf{X}^*$ based on the relative energy scree. That is, we retain the $K$ columns of $\mathbf{X}^*$ that have the smallest values of $\text{Scree}_k$ and set the remaining columns to $0$. We denote the thresholded version of $\mathbf{X}^*$ by $\widehat{\mathbf{X}}^{*(K)}$.
  \item \underline{\textbf{Apply IDWT to Each Row of the Thresholded Coefficient Matrix}}: To transform back to the data space, we apply the inverse DWT (IDWT) to each row of $\widehat{\mathbf{X}}^{*(K)}$. This will give use the reconstructed matrix $N\timesT_{\pad}$
  $$
  \widehat{\mathbf{X}}^{(K)}_{pad} = \text{IDWT}(\widehat{\mathbf{X}}^{*(K)}).
  $$
  To obtain a representation of the original signal length we simply discard the first $\lfloor (T_{pad} - T)/2 \rfloor$ columns and the last $\lceil (T_{pad} - T)/2 \rceil$ columns to give the matrix $\widehat{\mathbf{X}}^{(K)}$.
\end{steps}

\section{Additional Details on Squared Correlation Loss} \label{sec:squared-correlation}

\subsection{Complement of the Predicted Correlation Squared ($1-\rho^2$)}

Our loss measure is the complement of the squared correlation correlation among the observed data $X_i (\mathbf{t})$ and its predicted value $\widehat{X}^{(K)}_{i} (\mathbf{t})$:
$$
1- \rho^2 \left\{X_i (\mathbf{t}), \widehat{X}^{(K)}_{i} (\mathbf{t}) \right\} =
1 - \frac{\left[ \mathlarger{\sum}_{t = 1}^T{\bigg\{X_i (t) - \overline{X}_i \bigg\} \bigg\{ \widehat{X}_i^{(K)} (t) - \overline{\widehat{X}}_i^{(K)} \bigg\}} \right]^2}{\mathlarger{\sum}_{t = 1}^T \bigg\{X_i (t) - \overline{X}_i \bigg\}^2 \mathlarger{\sum}_{t = 1}^T \bigg\{ \widehat{X}_i^{(K)} (t) - \overline{\widehat{X}}_i^{(K)} \bigg\} ^2},
$$
where
$$
\overline{X}_i = \frac{1}{N} \sum_{t=1}^T X_i (t) \quad \text{and} \quad \overline{\widehat{X}}_i^{(K)} = \frac{1}{N} \sum_{t=1}^T \widehat{X}_i^{(K)} (t).
$$

\subsection{Predicted Residual Sum of Squares (PRESS)}

The predicted residual sum of squares (PRESS) statistics measures the discrepancy, in terms of total squared error, between the observed data $X_i (\mathbf{t})$ and its predicted value $\widehat{X}^{(K)}_{i} (\mathbf{t})$. 
The PRESS statistic is defined as the squared Euclidean distance between the observed data and its predicted value:
\begin{align*}
    \text{PRESS}\left\{X_i (\mathbf{t}), \widehat{X}^{(K)}_{i} (\mathbf{t}) \right\} 
&= 
\sum_{t = 1}^T \left\{ X_{i}(t) - \widehat{X}^{(K)}_{i} (t)\right\}^2 \\
&=
\bigg\| X_i (\mathbf{t}) - \widehat{X}^{(K)}_{i} (\mathbf{t}) \bigg\|^2\\
&=
\left\{X_i (\mathbf{t}) - \widehat{X}^{(K)}_{i} (\mathbf{t})\right\}^\top \left\{X_i (\mathbf{t}) - \widehat{X}^{(K)}_{i} (\mathbf{t})\right\}.
\end{align*}
While we have defined the PRESS statistic for individual observations, given our focus on individual information loss values, the total PRESS statistic summed over all observations is typically used to summarize the information loss in a PCA representation \parencite{bro_cross-validation_2008}.

\section*{Proof: Relationship between PRESS and $1 -\rho^2$ for PCA-based Projections}

When PCA is employed, $\widehat{X}^{(K)}_{i}(\mathbf{t})$ represents the projection of $X_{i}(\mathbf{t})$ onto \( K \)-dimensional subspace spanned by the first $K$ eigenvectors from PCA. That is
$$
 \widehat{X}^{(K)}_{i} (\mathbf{t}) = \underbrace{\boldsymbol{\Phi}_K}_{\text{Eigenvectors}} \underbrace{\boldsymbol{\Phi}_K^\top X_i(\mathbf{t})}_{\text{PC Scores}}
 = \mathbf{P} X_i(\mathbf{t}).
$$
where \( \mathbf{P} = \bm{\Phi}_K \bm{\Phi}_K^\top \) is the projection matrix, and \( \bm{\Phi}_K \in \mathbb{R}^{p \times K} \) satisfies \( \bm{\Phi}_K^\top \bm{\Phi}_K = \mathbf{I}_K \) (i.e., the eigenvectors, by definition, are orthogonal).
In the case where both $X_i(\mathbf{t})$ and $\widehat{X}^{(K)}_{i}(\mathbf{t})$ are mean-centered, i.e.,
$$
    X_i(\mathbf{t}) = \mathbf{A} X_i(\mathbf{t})
 \quad \textbf{and} \quad 
 \widehat{X}^{(K)}_{i}(\mathbf{t}) = \mathbf{A}\widehat{X}^{(K)}_{i}(\mathbf{t})
$$
for the centering matrix \( \mathbf{A} = \mathbf{I} - \frac{1}{T} \mathbf{1} \mathbf{1}^\top \), we can expand PRESS and $\rho^2$ to get
\begin{align*}
    \text{PRESS}\bigg\{X_{i}(\mathbf{t}),  \widehat{X}^{(K)}_{i} (\mathbf{t})\bigg\} &= \bigg\{X_{i}(\mathbf{t}) -  \widehat{X}^{(K)}_{i} (\mathbf{t})\bigg\}^\top \bigg\{X_{i}(\mathbf{t}) -  \widehat{X}^{(K)}_{i} (\mathbf{t})\bigg\} \\
    &= 
    X_{i}(\mathbf{t})^\top X_{i}(\mathbf{t}) - 2 X_{i}(\mathbf{t})^\top  \widehat{X}^{(K)}_{i} (\mathbf{t}) + \widehat{X}^{(K)\top}_{i\cdot}  \widehat{X}^{(K)}_{i} (\mathbf{t}) \\
    &= X_{i}(\mathbf{t})^\top X_{i}(\mathbf{t}) - 2 X_{i}(\mathbf{t})^\top  \widehat{X}^{(K)}_{i} (\mathbf{t}) + X_{i}(\mathbf{t})^\top \boldsymbol{\Phi}_K \underbrace{\boldsymbol{\Phi}_K^\top \boldsymbol{\Phi}_K}_{= \mathbf{I}_K} \boldsymbol{\Phi}_K^\top X_{i}(\mathbf{t}) \\
    &=  X_{i}(\mathbf{t})^\top X_{i}(\mathbf{t}) - 2 X_{i}(\mathbf{t})^\top  \widehat{X}^{(K)}_{i} (\mathbf{t}) + X_{i}(\mathbf{t})^\top \underbrace{\boldsymbol{\Phi}_K \boldsymbol{\Phi}_K^\top X_{i}(\mathbf{t})}_{= \widehat{X}^{(K)}_{i} (\mathbf{t})} \\
    &= X_{i}(\mathbf{t})^\top X_{i}(\mathbf{t}) - 2 X_{i}(\mathbf{t})^\top  \widehat{X}^{(K)}_{i} (\mathbf{t}) + X_{i}(\mathbf{t})^\top  \widehat{X}^{(K)}_{i} (\mathbf{t}) \\
    &= X_{i}(\mathbf{t})^\top X_{i}(\mathbf{t}) - X_{i}(\mathbf{t})^\top  \widehat{X}^{(K)}_{i} (\mathbf{t}).
\end{align*}
Likewise, we have 
\begin{align*}
    \rho^2  \bigg\{X_{i}(\mathbf{t}),  \widehat{X}^{(K)}_{i} (\mathbf{t})\bigg\}
    =&
    \bigg\{\widehat{X}^{(K)}(\mathbf{t})X_{i}(\mathbf{t}) X_{i}(\mathbf{t})^\top \widehat{X}^{(K)}(\mathbf{t}) \bigg\}
     \bigg\{X_{i}(\mathbf{t})^\top X_{i}(\mathbf{t}) \underbrace{\widehat{X}^{(K)\top}(\mathbf{t}) \widehat{X}^{(K)}_{i} (\mathbf{t})}_{=X_{i}(\mathbf{t})^\top  \widehat{X}^{(K)}_{i} (\mathbf{t})} \bigg\}^{-1} \\
     =& 
     \bigg\{\widehat{X}^{(K)^\top}(\mathbf{t})X_{i}(\mathbf{t}) X_{i}(\mathbf{t})^\top \widehat{X}^{(K)}(\mathbf{t}) \bigg\}
     \bigg\{X_{i}(\mathbf{t})^\top X_{i}(\mathbf{t}) X_{i}(\mathbf{t})^\top  \widehat{X}^{(K)}_{i} (\mathbf{t}) \bigg\}^{-1} \\
     =&
     \widehat{X}^{(K)^\top}(\mathbf{t})X_{i}(\mathbf{t}) \underbrace{X_{i}(\mathbf{t})^\top \widehat{X}^{(K)}(\mathbf{t})
     \bigg\{X_{i}(\mathbf{t})^\top \widehat{X}^{(K)}(\mathbf{t}) \bigg\}^{-1}}_{=\mathbf{I}}
     \bigg\{X_{i}(\mathbf{t})^\top X_{i}(\mathbf{t})\bigg\}^{-1}  \\
     =& \widehat{X}^{(K)^\top}(\mathbf{t})X_{i}(\mathbf{t})  \bigg\{X_{i}(\mathbf{t})^\top X_{i}(\mathbf{t})\bigg\}^{-1} \\
     =&
     \widehat{X}^{(K)}(\mathbf{t})^\top X_{i}(\mathbf{t})  \bigg\{X_{i}(\mathbf{t})^\top X_{i}(\mathbf{t})\bigg\}^{-1} + 
     \bigg\{X_{i}(\mathbf{t})^\top X_{i}(\mathbf{t})\bigg\}  \bigg\{X_{i}(\mathbf{t})^\top X_{i}(\mathbf{t})\bigg\}^{-1} \\ &- 
     \bigg\{X_{i}(\mathbf{t})^\top X_{i}(\mathbf{t})\bigg\}  \bigg\{X_{i}(\mathbf{t})^\top X_{i}(\mathbf{t})\bigg\}^{-1} \\
     =& 
     1 - \underbrace{\bigg\{ X_{i}(\mathbf{t})^\top X_{i}(\mathbf{t}) - \widehat{X}^{(K)}(\mathbf{t})^\top X_{i}(\mathbf{t}) \bigg\}}_{= \text{PRESS}\bigg\{X_{i}(\mathbf{t}),  \widehat{X}^{(K)}_{i} (\mathbf{t})\bigg\}} \bigg\{X_{i}(\mathbf{t})^\top X_{i}(\mathbf{t})\bigg\}^{-1} \\
     =&
     1 - \text{PRESS}\bigg\{X_{i}(\mathbf{t}),  \widehat{X}^{(K)}_{i} (\mathbf{t})\bigg\} \bigg\{X_{i}(\mathbf{t})^\top X_{i}(\mathbf{t})\bigg\}^{-1}.
\end{align*}
Thus, when we use the complement of the squared correlation as our loss, we have
\begin{align*}
    \text{Loss} \left\{ X_i(\mathbf{t}) \right\}  &= 1 - \rho^2  \bigg\{X_{i}(\mathbf{t}),  \widehat{X}^{(K)}_{i} (\mathbf{t})\bigg\} \\
        &= \text{PRESS}\bigg\{X_{i}(\mathbf{t}),  \widehat{X}^{(K)}_{i} (\mathbf{t})\bigg\} \bigg\{X_{i}(\mathbf{t})^\top X_{i}(\mathbf{t})\bigg\}^{-1} \\
    &=
    \frac{\text{PRESS}\bigg\{X_{i}(\mathbf{t}),  \widehat{X}^{(K)}_{i} (\mathbf{t})\bigg\}}{\sum_{t=1}^T X_i(t)^2} = \frac{\text{PRESS}\bigg\{X_{i}(\mathbf{t}),  \widehat{X}^{(K)}_{i} (\mathbf{t})\bigg\}}{\| X_i(\mathbf{t})\|^2},
\end{align*}
which is the PRESS statistic, normalized by the squared Euclidean norm of the vector $X_i(\mathbf{t})$.
While an analogous relationship does not hold exactly when $X_i(t)$ and $\widehat{X}_i(\mathbf{t})$ are not centered, because of the non-commutativity between the centering matrix $\mathbf{A}$ and the projection matrix $\mathbf{P}$, it provides us with an intuition for $1-\rho^2$ as a measure of distance between the observed data and its predictions, that is normalized to account for the scale of the data.


\section{Pseudocode for \texttt{GLaRe()}}\label{sec:algo-extra}

Algorithm \ref{alg:glare} provides pseudocode for the internal structure underlying the \texttt{GLaRe()} function.

\begin{algorithm}
\caption{Internal structure of the \texttt{GLaRe()} function.}\label{alg:glare}
\KwData{
\begin{itemize}
    \item \underline{Data matrix}: $\mathbf{X}$.
    \item \underline{Learning function}: \texttt{learn($x$, k)}.
    \item \underline{Loss function}: \texttt{loss($x$, $\hat{x}$)}.
    \item \underline{Latent dimensions sequence}: $K_{from}$, $K_{to}$, $K_{by}$ with length $K_{len}$.
    \item \underline{Number of folds for cross-validation}: \texttt{num\_fold}
\end{itemize}}
\KwResult{
\begin{itemize}
    \item \underline{Training Loss}: \texttt{Training\_Loss} ($N \times K_{len}$ matrix)
    \item \underline{Validation Loss}: \texttt{Validation\_Loss} ($N \times K_{len}$ matrix)
\end{itemize}
}

\For{$K$ from $K_{from}$ to $K_{to}$ by $K_{by}$ \Comment*[r]{Training}}{
    \texttt{full\_model} \gets \texttt{learn($x$ = $\mathbf{X}$, k = $K$)};
    
    $\mathbf{X}^*$ \gets \texttt{full\_model}.\texttt{Encode($\mathbf{X}$)};

    $\widehat{\mathbf{X}}$ \gets \texttt{full\_model}.\texttt{Decode($\mathbf{X}^*$)};
    
    \For{i from $1$ to $N$}{
        \texttt{Training\_Loss[$i$, $K$]} \gets \texttt{loss}($x = X_{i}, \hat{x} = \widehat{X}_{i}$);
        }
    }


    Divide $i=1,\dots,N$ randomly into \texttt{num\_fold} folds \Comment*[r]{Cross-Validation}
    
\For{$j$ from $1$ to \texttt{num\_fold}} {

    $\mathbf{X}_{\text{Training}} \gets \mathbf{X}[i \text{ not in fold }j,]$;

    $\mathbf{X}_{\text{Validation}} \gets \mathbf{X}[i \text{ in fold }j,]$;

    \For{$K$ from $K_{from}$ to $K_{to}$ by $K_{by}$}{
        
        \texttt{cv\_model} \gets \texttt{learn($x$ = $\mathbf{X}_{\text{Training}}$, k = $K$)};

        $\mathbf{X}^*_{\text{Validation}}$ \gets \texttt{cv\_model}.\texttt{Encode($\mathbf{X}_{\text{Validation}}$)};

        $\widehat{\mathbf{X}}$_{\text{Validation}} \gets \texttt{cv\_model}.\texttt{Decode($\mathbf{X}^*_{\text{Validation}}$)};
        
        \For{$i$ in fold $j$}{
            \texttt{Validation\_Loss[$i$, $K$]} \gets \texttt{loss}($x$ = $X_{i}$, $\hat{x}$ = $\widehat{X}_{\text{Validation}, i}$);
            }
        }
    }  
    Summarise and Return \texttt{Training\_Loss} and \texttt{Validation\_Loss};
    \jsm{[DOUBLE CHECK FORMATTING IN ALGORITHM -- SEEMS TO BE SOME SYNTAX ERROR]}
\end{algorithm}
\jsm{[INSERT SUPPLEMENTAL SECTION INSTRUCTING USER ON HOW TO DEFINE THEIR OWN LEARNING FUNCTION FOR A DESIRED LATENT FEATURE APPROACH -- WHAT SUBFUNCTIONS ARE REQUIRED, AND WHAT INPUTS/OUTPUTS -- IN ENOUGH DETAIL FOR THEM TO MAKE SOMETHING THAT PLUGS INTO THE GLARE FUNCTION]}
\section{Additional Results of Case Studies}\label{sec:additional-results}


\clearpage
\printbibliography
\end{document}
